\documentclass[11pt]{article}

\usepackage[top=50mm, bottom=50mm, left=50mm, right=50mm]{geometry}
%\usepackage[a4paper,top=2.5cm,bottom=2.5cm,left=2.5cm,right=2.5cm,marginparwidth=1.75cm]{geometry}
% for adding numbers for lines 
\usepackage{lineno}
\usepackage{amssymb}
\usepackage{amsmath}
\usepackage{amsthm}
\usepackage{epsfig}
\usepackage{caption}
\usepackage{subcaption}
\usepackage{graphics}
\usepackage{graphicx}
\usepackage{multirow}
\usepackage{multicol}
\usepackage{float}
\usepackage{multirow}
\usepackage{color}
\usepackage{lineno}
\usepackage{fullpage}
\usepackage{hyperref}
\usepackage[normalem]{ulem} 
\usepackage{makeidx}
\usepackage{xspace}
\usepackage{wrapfig}
\usepackage{newfloat}
\makeindex

\DeclareFloatingEnvironment[name={Supplementary Figure},fileext=lof]{suppfigure}


\newtheorem{theorem}{Theorem}
\newtheorem{Definition}{Definition}
\newtheorem{corollary}{Corollary}
\newtheorem{Theorem}{Theorem}
\newtheorem{Lemma}{Lemma}
\newtheorem{Claim}{Claim}
\newtheorem{Notation}{Notation}
\newtheorem{Algorithm}{Algorithm}
\newtheorem{Observation}{Observation}


%%%%%%%%%%%%%%% macros by yusu %%%%%%%%%%%%%
\newtheorem{lemma}{Lemma}
\newtheorem{claim}{Claim}
\newtheorem{obs}{Observation}

\definecolor{darkred}{rgb}{1, 0.1, 0.3}
\definecolor{darkblue}{rgb}{0.1, 0.1, 1}
\definecolor{darkgreen}{rgb}{0,0.6,0.5}
\newcommand{\yusu}[1]		{{ \textcolor{darkblue} {{\sc Yusu Says:} #1}}}

\newcommand {\mm}[1] {\ifmmode{#1}\else{\mbox{\(#1\)}}\fi}
\newcommand{\denselist}{\itemsep 0pt\parsep=1pt\partopsep 0pt}

\newcommand{\etal}		{{et al.}\xspace}
\newcommand{\myeg}	{{e.g.}\xspace}

\newcommand{\eps}		{\varepsilon}
\newcommand{\myanceq}	{{\succeq}} %ancerster 
\newcommand{\myanc}	{{\succ}} %ancerster 
\newcommand{\mydesceq} {{\preceq}}
\newcommand{\mydesc}		{{\prec}}
\newcommand{\dgh}		{\delta_{\mathcal{GH}}} %GH distance
\newcommand{\optdI}	{\mu}
\newcommand{\myeps}	{\eps}
\newcommand{\mygood}		{{good}\xspace}
\newcommand{\myk}		{\tau} %{\kappa}
\newcommand{\mydelta}	{\delta}
\newcommand{\nTone}	{{\widehat{T}_{1, \mydelta}^{\hat{f}}}}
\newcommand{\nTtwo}	{{\widehat{T}_{2, \mydelta}^{\hat{g}}}}
\newcommand{\Img}		{{\mathrm{Im}}}
\newcommand{\myL}		{{L}}
\newcommand{\superlevel}		{{super-level}\xspace}
\newcommand{\myC}		{{\mathrm{C}}}
\newcommand{\setSL}	{{\mathcal{L}}}
\newcommand{\slone}[1] 	{{\mathrm{L}^{(1)}_{#1}}}
\newcommand{\sltwo}[1] 	{{\mathrm{L}^{(2)}_{#1}}}
\newcommand{\hatT}		{{\widehat{T}}}
\newcommand{\levelC}		{\mathrm{Ch}} %{{levelC}}
\newcommand{\LCA}		{{LCA}}%common ancester
\newcommand{\DPalg}	{{\sf DPgoodmap}}
\newcommand{\modifyDPalg}	{{\sf modified-DP}}
\newcommand{\myF}		{{\mathcal{F}}}
\newcommand{\mypartial}[1]		{{partial-{#1}-good}\xspace}
\newcommand{\myvalue}		{\mathrm{value}}
\newcommand{\degbound}		{{degree-bound}\xspace}
\newcommand{\mydepth}		{{depth}\xspace}
\newcommand{\ep}		{{\varepsilon_p}}

\newcommand{\Intdopt}		{{\mydelta^*}}
\newcommand{\criSet}		{{\Pi}}
\newcommand{\Tcal}			{\mathcal{T}}
\newcommand{\mycost}		{{\mathrm{cost}}}
\newcommand{\uTone}		{{|T_1^f|}}
\newcommand{\uTwo}		{{|T_2^g|}}
\newcommand{\uT}				{{|T|}}
\newcommand{\mywF}		{{w^F}}
\newcommand{\mykpara}	{{\myk_\mydelta}}
\newcommand{\newdepth}	{{depth}}
\newcommand{\newtau}	{{\widehat{\tau}}}

\newcommand{\sensiblepair}     {{sensible-pair}\xspace}
\newcommand{\Edgelistpair}      {{Edge-list pair}\xspace}
\newcommand{\edgelistpair}      {{edge-list pair}\xspace}
\newcommand{\Fnew}          {{F_{new}}}
\newcommand{\Fold}          {{F_{old}}}

%%%%%%%%%%%%%% macros by yusu end %%%%%%%%%%%%%

\begin{document}

\title{Monitoring Health Systems by Estimating Excess Mortality}
 
\author{
{Rolando J. Acosta \footnote{Department of Biostatistics, Harvard T.H. Chan School of Public, Boston, MA, USA}} \and {Rafael A. Irizarry \footnote{Department of Data Science, Dana-Farber Cancer Institute; \newline Department of Biostatistics, Harvard T.H. Chan School of Public, Boston, MA, USA}}
}

%\begin{document}
\maketitle
%\linenumbers
\setcounter{page}{0}

\begin{abstract}
Monitoring health systems during and after natural disasters, epidemics, or outbreaks is critical for guiding policy decisions and interventions. In the case of natural disasters the effects on public health can be \textit{direct} or \textit{indirect}. Direct effects are defined as those resulting from the immediate destruction such as drowning and trauma from flying debris, while indirect effects are delayed, longer lasting, and, often harder to measure or detect. In the case of outbreaks and epidemics, lack of comprehensive testing or reporting can lead to challenges in measuring direct effects, while indirect effects can arise due to, for example, increased stress levels or reduced access to health services. When the effects are long lasting and difficult to detect, in the short term, the accumulated effects can actually be devastating. Improved access to mortality data provides an opportunity to develop data-driven approaches that can help monitor health systems and quantify the effects of natural disasters, epidemics, or outbreaks. Here we describe a statistical methodology and software that facilitates data-driven approaches. Our work was motivated by events occurring in Puerto Rico after the passage of Hurricane Mar\'ia, but can be applied in other contexts such as estimating the effects of an epidemic in the presence of inaccurate reporting of cases. We demonstrate the utility of our tools by applying it to data related to six hurricanes and data related to the COVID-19 pandemic. We also searched for other unusual events during the last 35 years in Puerto Rico. We demonstrate that the effects of hurricanes in Puerto Rico are substantially worse than in other states, that the 2005 flu and the 2014 and the Chikungunya outbreak resulted in an unusually high mortality rate in Puerto Rico, that in the United States the excess mortality during the COVID-19 pandemic already exceeded 100,000 on May 9, 2020, and that the effects of this pandemic was worse for elderly black individuals compared to whites of the same age. We make our tools available through free and open source \verb+excessmort+ R package  and an interactive webtool available at XXXX.
\end{abstract}

\newpage

\section{Introduction}
\label{sec:introduction}
Hurricane Mar\'ia made landfall in Puerto Rico on September 20, 2017, interrupting the water supply, electricity, telecommunications networks, and access to medical care\cite{updates2018puerto, rogers2017puerto}. Two weeks after the hurricane, the official death count stood at 16\cite{hoyos2018puerto}. This figure was in conflict with estimates obtained using a historical approach by groups that ostensibly had access to death counts for September and October from the demographic registry of Puerto Rico, not yet readily available to the general public. By comparing these two numbers to historical averages, additional deaths were estimated to be in excess of 1,000 \cite{santos2018use, robles2017official, rivera2018estimating, pascual2017nearly} raising concerns that Puerto Rico was in the midst of a public health crisis. However, on December 13, 2017, almost three months after landfall, Puerto Rico’s Secretary of the Department of Public Safety, dismissed these concerns \cite{robles2017official}. As of May 2017 the official death count stood at 64 \cite{hoyos2018puerto} and it appeared the government was not concerned. On May 29, 2018 three days after a study was published describing a survey of 3,299 households and reporting a much higher estimate of the death count received worldwide media coverage \cite{kishore2018mortality}, the government finally made the data public and acknowledged the possibility of a higher excess death count of 1,427 \cite{report2019} based on the vital statistics numbers. On August 2018 the official death toll was revised to 2,975 based on a study commissioned by the government of Puerto Rico \cite{santos2018differential}.

More recently, the limited capacity to perform molecular tests has raised concerns that COVID-19 cases are under-reported in several jurisdictions in the United States and across the world. Simultaneously, polls show that a substantial proportion of the US population believes that reported COVID-19 fatalities are inflated \cite{talev2020}. Estimating excess mortality provides a way to measure the impact of the epidemic, even in the presence of inaccurate reporting.


Here we describe data-driven tools that, when applied to death records, can help answer questions such as: are current excess mortality counts a reason for concern? For how long is the population affected by an event?  How much do the effects vary across jurisdictions?  Was there a specific demographic that was at greater risk? Our approach provide informative summaries that can help detect concerning patterns, including tools that can also help detect abnormalities in close to real-time. 

The main goal of our method is to determine if an observed death count, for a given demographic and period of time is \textit{unusually high}. Reporting unusually high death rates can alert government officials and policy makers that an intervention is required. Because of stochastic variation, \textit{usual} cannot be defined with one number and instead a distribution is appropriate. The uncertainty associated with count data, such as daily deaths totals, are typically accounted for with Poisson models \cite{mccullagh19831989, agresti2015foundations}. However, these models do not account for year-to-year increases or decreases that may be due to natural, possibly correlated in time, stochastic variation. Examples of possible causes are variation in the severity of common colds in the winter or the number of extremely hot days in the summer. Ignoring this source of correlated variation can lead to underestimates of the uncertainty associated with mortality excess estimates. The engine behind our tools is a statistical model that accounts for such variability and can be fit to data stratified by demographic group or cause of death. 

To demonstrate the utility of our approach, we provide a detailed picture of the effect Hurricane Mar\'ia had on mortality in Puerto Rico and compare these to those observed in Hugo and Georges, two previous hurricanes in Puerto Rico, Katrina in Louisiana, Irma in Florida, and Sandy in New Jersey. We also used or approach to search for unusually high mortality rates during the last 35 years in Puerto Rico. We then apply our method to recently released state-level data by the CDC and compare the excess deaths during the COVID-19 epidemic to those from the 2018 flu season. We also compare the excess mortality between racial groups in Cook County, Illinois, the only county we are aware of that is currently releasing up to date medical examiner records. Finally, we propose a data-driven index that does not require complete data nor an estimate of population size, which is useful in cases in which estimating population size is challenging thus making real-time estimates less reliable. We make our method available through the \verb+excessmort+ R package and an interactive webtool. \\

\section{Results}
\label{sec:results}

\subsection{Model-based approach provides informative parameterizations}
\label{subsec:model-based}
Extending the work developed for monitoring outbreaks \cite{farrington1996statistical,hohle2008count,noufaily2013improved,salmon2016monitoring} we modeled death counts with the following mixed model
\begin{equation}
    Y_t \mid \varepsilon_t \sim 
    \mbox{Poisson}\left\{ \mu_t \right[1 + f(t) \left]  \varepsilon_t \right\} \mbox{ for } t = 1, \dots,T
\end{equation}
with $\mu_t$ the expected number of deaths at time $t$ for a \emph{typical period}, $100 \times f(t)$ the percent increase at time $t$ due to an unusual event, $\varepsilon_t$ a time series of auto-correlated random variables representing natural variability, and $T$ the total number of observations. The expected counts $\mu_t$ can be further decomposed into 
\begin{equation}
    \mu_t =N_t \exp[\alpha(t) + s(t) + w(t)]
    \label{eq:mean-model}
\end{equation}
with $N_t$ the population at time $t$, $\alpha(t)$ a slow moving trend that accounts for changes such as the improved health outcomes we have observed during the last 20 years (Supplemental Figure \ref{supp-fig:model-components}A), $s(t)$ a yearly periodic function representing a seasonal trend (Supplemental Figure \ref{supp-fig:model-components}B), and $w(t)$ is a day of the week effect. We refer to $f$ as the \emph{event effect} and assume $f(t) = 0$ for typical periods not affected by a natural disaster or outbreak. As described in more details in the Methods Section, our model permit non-continuities in $f(t)$ to account for sudden direct effects caused by natural disasters. Note that we can conveniently represent excess deaths at time $t$ as $\mu_t f(t)$ and compute excess deaths for a time period by just adding these up for the $t$'s in the interval of interest. 
Details are provided in the Methods Section.

\subsection{Detecting and quantifying indirect effects after hurricanes}
\label{subsec:effects}
Using our approach, we found that the death rate in Puerto Rico increased by 69\% (95\% CI: 56\% to 82\%) the days after hurricane Mar\'ia made landfall. The increased death rate persisted until March (Figure \ref{fig:fhat-estimates}A, Supplemental Figure \ref{supp-fig:hurricane-effects}). To understand if such a large and sustained increase is common after the passage of a hurricane, we examined mortality counts in Louisiana after Katrina, New Jersey after Sandy, Florida after Irma as well as two other hurricanes in Puerto Rico, Hugo in 1989 and Georges in 1998.  Although not as severe, a similar pattern to Mar\'ia was observed for Georges in 1998 when the death rate increased by 33\% (95\% CI: 22\% to 44\%) the days after landfall and an increased death rate persisted until December (Figure \ref{fig:fhat-estimates}A, Supplemental Figure \ref{supp-fig:hurricane-effects}). The effects of Katrina on Louisiana were much more direct. On August 29, 2005, the day the levees broke, there were 834 deaths, which translates to an increase in death rate of 689\% (not shown in Figure \ref{fig:fhat-estimates}). However, the increase in mortality rates for the months following this catastrophe was substantially lower than in Puerto Rico after Mar\'ia and Georges. None of the other hurricanes examined had noticeable direct or indirect effects (Figure \ref{fig:fhat-estimates}A, Supplemental Figure \ref{supp-fig:hurricane-effects}).

\begin{figure}[ht]
	\begin{tabular}{lll}
	A&B&C\\
	\begin{subfigure}[t]{0.30\linewidth}
		\centering
		\includegraphics[width=1\linewidth]{figs/figure-1a.pdf} 
	\end{subfigure}&
	\begin{subfigure}[t]{0.30\linewidth}
		\centering
		\includegraphics[width=1\linewidth]{figs/figure-1b.pdf}
	\end{subfigure}&
	\begin{subfigure}[t]{0.30\linewidth}
		\centering
		\includegraphics[width=1\linewidth]{figs/figure-1c.pdf}
	\end{subfigure}\\
    \end{tabular}
    \caption{Estimated event effects as percent increase over expected mortality. A) Comparison of six hurricanes. B) Estimated increase and point-wise confidence intervals among individuals 60 or over during the Chikungunya epidemic in Puerto Rico (Aug 2014 - Feb 2015) and the 2005 flu season (November 2004 - Apr 2005). C) Estimated increase and confidence interval for the the United States from January 2017 through May 2020.}
    \label{fig:fhat-estimates}
\end{figure}

\subsection{Quantifying the effects of epidemics}
\label{subsec:epidemics}
Our approach can also be used to detect and quantify the effects of epidemics or outbreaks. As an example, we fit our model to Puerto Rico mortality data from 1985 to 2020 and, apart from the hurricane seasons mentioned in the previous section, we detected two periods with an unusual increase in mortality rates: November 2004 - April 2005 and August 2014 - February 2015. These periods coincide with the 2005 flu season and the 2014-2015 Chikungunya outbreak \cite{sharp2016surveillance, hsu2019risk}. The effects were particularly strong on individuals over 60 years (Figure \ref{fig:fhat-estimates}B, Supplemental Figure \ref{supp-fig:chikungunya-agegroups}). Note that for the Chikungunya period we also detected a strong effect for the  0 to 4 years age group. 

Although mortality records for the time period of the COVID-19 pandemic are not yet complete, the Center for Disease Control and Prevention (CDC) has implement a correction to the most recent counts and are making weekly state-level data from January 2017 to May 12, 2020 publicly available \cite{cdc2020covid19}. We fit our model to these data for each state and our event effect estimates clearly detects the COVID-19 pandemic for several states, as well as the 2018 flu epidemics (Supplemental Figure \ref{supp-fig:covid19-states}). By aggregating the results for the entire United States we confirm that the COVID-19 pandemic had a substantially larger impact than the 2018 flu season
(Figure \ref{fig:fhat-estimates}C).
 
\subsection{Excess mortality estimates}
\label{subsec:excess-mortality}
Once we have fit our model, we can compute excess mortality estimates for any time period as well as standard errors that take into account, not just sample variability associated with count processes, but natural variation as well. We used the estimated parameters from the Puerto Rico data to compute excess deaths for the 365 days following the day hurricanes Georges, and Mar\'ia made landfall. We also computed these quantities for the 365 days following the start of the detected effects from the 2005 flu season and 2014 Chikungunya outbreak. Finally, we computed excess deaths in Puerto Rico during the COVID-19 pandemic, specifically from March 1, 2020 to April 15 2020, the last day for which the death registry has complete data (typically, it takes 90 days for 99\% of deaths to be registered). 

For Mar\'ia the excess death curve grew for 198 days after landfall, with a point estimate on April 6, 2018 of 3,284 (95\% CI: 2,896 to 3,673) (Figure \ref{fig:excess-deaths}A, Supplemental Figure \ref{supp-fig:excess-deaths}). For Georges the excess death curve grew for 96 days after landfall, with a point estimate on December 26, 1998 of 1,201 (95\% CI: 946 to 1,457)  (Figure \ref{fig:excess-deaths}A, Supplemental Figure \ref{supp-fig:excess-deaths}). For the period of October 10, 2004 to March 28, 2005, identified to have unusually high mortality rates and including the 2005 flu season, we observe a point estimate of 1,133 (95\% CI: 795 to 1,471) (Figure \ref{fig:excess-deaths}A, Supplemental Figure \ref{supp-fig:excess-deaths}). The period of August, 7, 2014 to February 28, 2015 associated with a Chikungunya outbreak was also identified to have unusually high mortality rates. However, we also noted a decrease in cumulative deaths from February 28, 2015 to May, 1, 2015 consistent with a \emph{harvesting effect} \cite{hajat2005mortality, dushoff2006mortality}. For this entire period, August, 7, 2014 to May, 2, 2015, we observe a point estimate for excess mortality of 1,308 (95\% CI: 877 to 1,739) (Figure \ref{fig:excess-deaths}A, Supplemental Figure \ref{supp-fig:excess-deaths}).

\begin{figure}[ht]
	\begin{tabular}{lll}
	A&B&C\\
	\begin{subfigure}[t]{0.3\linewidth}
		\centering
		\includegraphics[width=1\linewidth]{figs/figure-2a.pdf} 
	\end{subfigure}&
	\begin{subfigure}[t]{0.3\linewidth}
		\centering
		\includegraphics[width=1\linewidth]{figs/figure-2b.pdf}
	\end{subfigure}&
	\begin{subfigure}[t]{0.3\linewidth}
		\centering
		\includegraphics[width=1\linewidth]{figs/figure-2c.pdf}
	\end{subfigure}\\
	\end{tabular}
    \caption{Cumulative excess mortality estimates. A) Daily excess death estimates and 95\% confidence intervals for events in Puerto Rico starting on Sep 20, 2017, Sep 21, 1998, Aug 1, 2014, Mar 1, 2020, and Oct 1, 2004 for Hurricane Mar\'ia, Hurricane Georges, the Chikungunya epidemic, the COVID-19 pandemic, and the 2005 Flu season, respectively.  B) Estimated cumulative excess mortality and reported COVID-19 cumulative deaths in the United States. C) State-specific cumulative excess death estimates for COVID-19 plotted against the excess mortality estimates for the 2018 Flu season. The identity line is represented in red and separates states in which the increase in mortality was higher during COVID-19 from those in which the increase in mortality was higher during the 2018 flu season}
    \label{fig:excess-deaths}
\end{figure}

Next, using the CDC data we computed excess mortality for the period of the COVID-19 pandemic for which we have data, specifically for the week ending on March 15, 2020 to the week ending on May 9, 2020. When we compared these numbers to the reported numbers of COVID-19 deaths \cite{smith2020coronavirus}, we find a difference of 29,246 (95\% CI: 27,317 to 31,175) deaths (Figure \ref{fig:excess-deaths}B, \ref{tab:covid19-states}). We also note that in the US, the number of excess deaths during the COVID-19 pandemic are much larger than the 2018 flu season, although this is not true for all states (Figure \ref{fig:excess-deaths}C, Table \ref{tab:covid19-states}). We also note that 58\% of the cases come from the four northeast states: New York, New Jersey, Massachusetts, and Pennsylvania (Supplemental Figure \ref{supp-fig:covid19-states}).

\begin{table}
\centering
\small
\begin{tabular}{|l|lll|ll|}
  \hline
 & \multicolumn{3}{c|}{COVID-19} & \multicolumn{2}{c|}{2018 Flu} \\
\hline
State & Excess & Reported & Rate  & Excess & Rate \\
 & mortality (SD) & cases & (per million) & mortality (SD) & (per million) \\
\hline
New York City & 24470 (127) & 19324 & 2939 & 732 (122) & 88 \\ 
  New Jersey & 14537 (145) & 9116 & 1637 & 464 (139) & 52 \\ 
  New York& 11183 (169) & 7260 & 1011 & 1117 (165) & 101 \\ 
  Massachusetts & 6159 (124) & 4840 & 892 & 314 (119) & 45 \\ 
  Michigan & 6800 (179) & 4526 & 681 & 919 (177) & 92 \\ 
  Pennsylvania & 8383 (200) & 3793 & 655 & 1015 (192) & 79 \\ 
  Louisiana & 2971 (126) & 2267 & 640 & 1009 (122) & 218 \\ 
  District of Columbia & 441 (35) & 311 & 622 & 111 (34) & 156 \\ 
  Maryland & 2666 (114) & 1614 & 440 & 339 (111) & 56 \\ 
  Illinois & 5399 (194) & 3362 & 428 & 1601 (188) & 127 \\ 
  Delaware & 376 (46) & 221 & 383 & 38 (45) & 39 \\ 
  Rhode Island & 328 (47) & 418 & 309 & 81 (47) & 76 \\ 
  Mississippi & 913 (89) & 421 & 307 & 766 (87) & 258 \\ 
  Vermont & 186 (41) & 53 & 298 & 108 (40) & 173 \\ 
  Indiana & 1942 (136) & 1490 & 287 & 943 (133) & 139 \\ 
  Colorado & 1554 (86) & 965 & 267 & 299 (87) & 51 \\ 
  Virginia & 1891 (126) & 827 & 221 & 660 (124) & 77 \\ 
  Georgia & 2006 (190) & 1380 & 187 & 1247 (187) & 116 \\ 
  South Carolina & 898 (109) & 330 & 172 & 619 (107) & 119 \\ 
  Ohio & 1928 (221) & 1331 & 165 & 1809 (214) & 155 \\ 
  Arizona & 1193 (148) & 532 & 162 & 1419 (149) & 192 \\ 
  New Hampshire & 215 (51) & 131 & 158 & 28 (50) & 21 \\ 
  New Mexico & 289 (69) & 191 & 138 & 73 (69) & 35 \\ 
  Alabama & 652 (122) & 390 & 133 & 902 (119) & 184 \\ 
  Missouri & 813 (160) & 492 & 132 & 1104 (155) & 179 \\ 
  Florida & 2846 (352) & 1714 & 131 & 2320 (348) & 107 \\ 
  Washington & 975 (160) & 925 & 127 & 69 (156) & 9 \\ 
  Minnesota & 714 (113) & 558 & 126 & 469 (112) & 83 \\ 
  Wisconsin & 718 (126) & 398 & 123 & 642 (125) & 110 \\ 
  Texas & 2989 (349) & 1086 & 102 & 3890 (339) & 133 \\ 
  California & 4011 (475) & 2732 & 101 & 4636 (461) & 117 \\ 
  Nebraska & 151 (63) & 95 & 78 & 70 (61) & 36 \\ 
  Iowa & 221 (87) & 252 & 70 & 126 (85) & 40 \\ 
  Nevada & 206 (72) & 306 & 66 & 341 (72) & 109 \\ 
  Tennessee & 439 (163) & 241 & 64 & 1061 (160) & 154 \\ 
  Wyoming & 28 (29) & 7 & 48 & 10 (29) & 17 \\ 
  Kentucky & 202 (107) & 312 & 45 & 723 (105) & 162 \\ 
  Utah & 124 (58) & 66 & 38 & 64 (59) & 20 \\ 
  Oregon & 156 (106) & 127 & 37 & 122 (105) & 29 \\ 
  West Virginia & 36 (72) & 53 & 20 & 503 (71) & 283 \\ 
  Idaho & 33 (50) & 67 & 18 & 83 (51) & 46 \\ 
  Arkansas & 49 (89) & 88 & 16 & 526 (87) & 174 \\ 
  Montana & 16 (46) & 16 & 15 & -80 (45) & -75 \\ 
  Kansas & 33 (77) & 172 & 11 & 172 (75) & 59 \\ 
  Maine & 1 (51) & 64 & 1 & 44 (52) & 33 \\ 
  South Dakota & -11 (45) & 34 & -13 & -40 (44) & -45 \\ 
  Hawaii & -83 (48) & 17 & -59 & 88 (48) & 62 \\ 
  Alaska & -44 (30) & 8 & -61 & -4 (31) & -5 \\ 
  Oklahoma & -424 (111) & 270 & -107 & 582 (107) & 147 \\ 
  North Dakota & -123 (39) & 35 & -160 & 32 (40) & 41 \\ 
 \hline
  Total & 111456 (1041) & 75228 & 353 & 34167 (1017) & 108 \\ 
  \hline
 \end{tabular}
 \caption{Excess mortality estimates for the weeks ending on March 14, 2020 to May 5, 2020 (COVID-19) and  December 16, 2017 to February 10, 2018 (2018 Flu). The standard deviation for excess mortality for these periods is included in parenthesis. The states are ordered by the per-million excess mortality rate. Connecticut and North Carolina are excluded due to missing data. Numbers for New York do not include New York City which is listed separately.}
 \label{tab:covid19-states}
 \normalsize
\end{table}
 
 

\subsection{Across group comparisons}
\label{subsec:group-comparison}
Since our model provides both estimates and standard errors of event effects, we can compare the effects for two or more groups and assess the statistical significance of the observed differences. We demonstrate how this can be used by comparing black and white populations during the COVID-19 pandemic in Cook County, Illinois. At the time of writing, the Cook County Medical Examiner's Office was making up-to-date records available online including demographic information for each individual. This office investigates any human death in Cook County that falls within categories such as criminal violence, suicide, accident, diseases constituting a threat to public health, among other unusual circumstances \cite{cookcovid19}. If we consider $\mu_t$ to be the expected number of deaths recorded by the Medical Examiner's Office, we can fit our model and interpret $f(t)$ as the percent increase in deaths examined by this office. We fit the model to these data and clearly observed the effects of the COVID-19 epidemic, in particular among individuals over 75,  however, a persistent increase in mortality above normal levels is observed for individuals above 40 (Supplemental Figure \ref{supp-fig:cook-agegroups}).  Furthermore, we found a difference in the effect  between white and black individuals in the months of March and April (Figure \ref{fig:cook-white-v-black}). Among those between 40 and 59 years of age, the percent increase in medical examiner death rates for Cook county was as high as 169\% (95\% CI: 134\% to 203\%) and 85\%  (95\% CI: 54\% to 115\%) on April 14 for blacks and whites, respectively. Furthermore, for the 60 to 74 age group we observed a peak percent increase of 630\% (95\% CI: 544\% to 716\%) and 497\% (95\% CI: 424\% to 571\%), respectively. Finally, the percent increase for individuals 75 years and older was 2,334\% (95\% CI: 2,056\% to 2,612\%) and 1,213\% (95\% CI: 1,092\% to 1,334\%), respectively, another statistically significant finding (Figure \ref{fig:cook-white-v-black}).  The population size for other racial groups was substantially smaller and were not included in the analysis.

\begin{figure}[ht]
	\begin{tabular}{lll}
	A&B&C\\
	\begin{subfigure}[t]{0.3\linewidth}
		\centering
		\includegraphics[width=1\linewidth]{figs/figure-3a.pdf} 
	\end{subfigure}&
	\begin{subfigure}[t]{0.3\linewidth}
		\centering
		\includegraphics[width=1\linewidth]{figs/figure-3b.pdf}
	\end{subfigure}&
	\begin{subfigure}[t]{0.3\linewidth}
		\centering
		\includegraphics[width=1\linewidth]{figs/figure-3c.pdf}
	\end{subfigure}\\
	\end{tabular}
	\caption{Comparison of estimated percent increase in mortality in Cook county, Illinois, for individuals above 40 between whites and blacks with corresponding 95\% confidence intervals.}
	\label{fig:cook-white-v-black}
\end{figure}


\subsection{An index based on relative proportion of specific causes of death}
\label{subsec:causes-deaths}

Limitations of our approaches for detecting and monitoring health crises in real-time are that it requires 1) complete death count data and 2) an estimate of population displacement. Note that incomplete death counts or underestimates of population displacement will result in an underestimate of the death rate. However, if we redefine $Y_t$ as the number of deaths due to a specific cause at time $t$ and $N_t$ as the total number of recorded deaths at time $t$, then the estimate of $f$ can be interpreted as the percent increase in the proportion of deaths due to that cause, even if the data is incomplete. The underlying cause of death for the Puerto Rico mortality data was coded using the tenth version of the International Classifications of Diseases (ICD10) codes \cite{world2004icd}. Because deaths due to bacterial infections have been reported to increase after natural disasters in locations with poor public health infrastructure \cite{ligon2006infectious, cook200810th}, we defined an index based on ICD10 codes between A00 and A79 as bacterial infections. We found that deaths attributable to this cause were elevated after the landfall of hurricane Mar\'ia in Puerto Rico (Figure \ref{fig:mortality-index}A) pointing the possibility that this index could be used in real-time to detect challenges to the health system. During the COVID-19 pandemic, if under-reporting is occurring, one would expect causes related to the respiratory system to increase. A similar index for respiratory diseases defined as ICD10 codes between J00 and J99 computed for the COVID-19 pandemic shows no anomalies  (Figure \ref{fig:mortality-index}B).

\begin{figure}[ht]
	\begin{tabular}{lll}
	A&B\\
	\begin{subfigure}[t]{0.49\linewidth}
		\centering
		\includegraphics[width=1\linewidth]{figs/figure-4a.pdf} 
	\end{subfigure}&
	\begin{subfigure}[t]{0.49\linewidth}
		\centering
		\includegraphics[width=1\linewidth]{figs/figure-4b.pdf}
	\end{subfigure}&\\
    \end{tabular}
	\caption{Cause-specific mortality index. The solid-black line corresponds to the percent change in proportion of mortality, $\hat{f}(t)$, attributable to a cause of interest. The shaded area represents a 95\% confidence interval for $\hat{f}(t)$. A) Bacterial infections mortality index defined as ICD10 codes between A00 and A79 for the days around Hurricane Mar\'ia. B) Respiratory diseases mortality index defined as ICD10 codes between J00 and J99 for the dates around the COVID-19 pandemic in Puerto Rico.}
	\label{fig:mortality-index}
\end{figure}



\subsection{Natural Variability and Correlated Counts}
\label{subsec:natural-variability}
Exploration data analysis after fitting a standard Poisson model demonstrates that the daily data measurements are correlated (Supplemental Figure \ref{supp-fig:correlated-errors}). Assuming independence, when in fact variables are correlated, leads to underestimates of the standard error of sums or averages. Cumulative excess mortality is the sum of excess deaths during a period of interest, therefore accounting for correlation allows our approach to better estimate the uncertainty of this quantity. To demonstrate this, we compared our model to a standard Poisson model and a over-dispersed Poisson model both assuming independent outcomes \cite{farrington1996statistical,hohle2008count,noufaily2013improved,salmon2016monitoring}. We fit each of the three models to the Puerto Rico data and randomly selected 100 intervals of sizes $L$ = 10, 50, and 100 days from periods with no events and computed the total number of deaths in each interval $S_l = \sum_{t \in L} Y_t$. We then used the fitted models to estimate the expected value $\hat{\mbox{E}}(S_l)$ and standard error  and $\hat{\mbox{SE}}(S_l)$ and form a z-statistic $Z_l = [S_l - \hat{\mbox{E}}(S_l)]/\hat{\mbox{SE}}(S_l)$. The Central Limit Theorem predicts that, if the expected value and standard errors are estimated correctly, $Z_l$ is approximately normal with expected value 0 and standard error 1. We found that the Poisson and over-dispersed Poisson model underestimate the variance of this sum, while modeling the correction improves our standard error estimates (Figure \ref{fig:deaths-qqplots}). 

\begin{figure}[ht]
	\begin{tabular}{lll}
	A&B&C\\
	\begin{subfigure}[t]{0.3\linewidth}
		\centering
		\includegraphics[width=1\linewidth]{figs/figure-5a.pdf} 
	\end{subfigure}&
	\begin{subfigure}[t]{0.3\linewidth}
		\centering
		\includegraphics[width=1\linewidth]{figs/figure-5b.pdf}
	\end{subfigure}&
	\begin{subfigure}[t]{0.3\linewidth}
		\centering
		\includegraphics[width=1\linewidth]{figs/figure-5c.pdf}
	\end{subfigure}\\
	\end{tabular}
	\caption{Accounting for correlation in the error structure improves uncertainty estimates. We fitted our model along with a Poisson and over-dispersed Poisson GLM to Puerto Rico data and randomly picked 100 intervals of varying sizes to compute z-scores of total number of deaths. Each plot shows the Pearson residual quantiles versus theoretical quantiles from the standard normal distribution. A) Intervals of 10 days. B) Intervals of 50 days. C) Intervals of 100 days.}
	\label{fig:deaths-qqplots}
\end{figure}

\section{Methods}
\label{sec:methods}

In this section we provided a description of the model and estimation procedure applied to obtain results. All details of our analysis are available by examining the code which is made available on GitHub: https://github.com/rafalab/excess-mortality-paper

\subsection{Mortality Data}
\label{subsec:mortality-data}
We obtained detailed mortality and population size data related to hurricanes Hugo, Georges, and Mar\'ia in Puerto Rico, Katrina in Louisiana, Sandy in New Jersey, and Irma in Florida. For Puerto Rico, we requested individual level mortality information with no personal identifiers from the Department of Health of Puerto Rico and obtained individual records including date, gender, age, and cause of death from January 1985 to May 2020. We used this data to construct daily death counts for different strata of the Puerto Rican population throughout the study period. We also obtained daily death counts from Florida, New Jersey, and Louisiana’s Vital Statistic systems and obtained data from January 2015 to December 2018, January 2007 to December 2015, and January 2003 to December 2006, respectively. Note that we did not obtain individual records from these jurisdictions. We also obtained mortality data made public on May 29, 2020 by the CDC  \cite{cdc2020covid19}. This dataset included weekly estimates of death counts for each state and Puerto Rico for the period of January 2017 and May 2020. 
Finally, we obtained individual level mortality data from the Cook County, IL, medical examiner office \cite{cookcovid19}. These data included demographic information for each death including age, sex, and race.

\subsection{Population Estimates}
\label{subsec:population}
Yearly population estimates for Puerto Rico were obtained from the Puerto Rico Statistical Institute. Yearly population estimates for other US states were obtained from the US Census bureau. We computed daily population estimates via linear interpolation of the available data. To obtain an estimate of the population displacement after Hurricane Mar\'ia in Puerto Rico, we use population proportion estimates from Teralytics. Teralytics is a tech-company that works with governments and private clients to assess human movement by partnering with mobile network operators. From them we obtained island-wide daily population proportion estimates relative to an undisclosed baseline from May 31, 2017 to April 30, 2018 on the basis of all subscribers of a major, undisclosed, telecommunications company that created events in Puerto Rico. Due to structural damage post Hurricane Maria proportion estimates for four weeks after the storm were not provided. We imputed missing data with linear interpolation of observed values, computed population estimates by multiplying the US census population estimate on July 19, 2017, Teralytics baseline date, times the aforementioned proportions, and obtained smooth estimates by fitting a \textit{loess} regression (Supplemental Figure \ref{supp-fig:pop-displacement-pr}).

\subsection{Statistical model}
\label{subsec:statistical-model}
We assume the mortality counts at $t$ follows a generalized linear mixed model defined above, in section \ref{subsec:model-based}. Specifically, we assume:
\begin{equation*}
    Y_t \mid \varepsilon_t \propto \mbox{Poisson}\left\{ \mu_t 
    \left[1 + f(t)\right]  \varepsilon_t \right\}, \mbox{ for } t = 1, \dots,T 
\end{equation*}
where the expected counts have the form:
\begin{equation*}
    \mu_t =N_t  \exp[\alpha(t) + s(t) + w(t)]
\end{equation*}
and the vector $[\varepsilon_1, \ldots, \varepsilon_T]^\top$ follows a multivariate normal distribution with mean 1 and variance-covariance matrix $\Sigma$ determined by an auto-regressive (AR) process of order $p$:
\begin{equation*}
    \Sigma = \sigma^2
    \begin{pmatrix} 1 & \rho_1 & \rho_{2} & \dots & \rho_{T} \\ 
    \rho_{1} & 1 & \rho_{1} & \ldots & \rho_{T-1} \\
    \vdots & \vdots & \vdots & \vdots & \vdots \\
    \rho_{T} & \rho_{T-1} & \rho_{T-2} & \dots & 1 \\ 
    \end{pmatrix}.
\end{equation*}
Here $\sigma$ corresponds to the standard deviation associated with natural variability and $\rho_h$ is the auto-correlation of two error terms that are $h$ time units from each other. In practice, obtaining Maximum Likelihood Estimates (MLE) for this model is not straight-forward, in particular because the flexibility in $f(t)$ and $\Sigma$ makes the model non-identifiable in practice. To overcome this challenge, we implemented a three step approach that works well in practice, as demonstrated both by simulation and empirical validation.

We start by modeling the yearly trend $\alpha(t)$ with natural cubic splines with a knot every seven years. We model the seasonal effect $s(t)$ with a harmonic model with 2 harmonics. Finally, $w(t)$
represents seven parameters, one for each day of the week. 

Natural disasters and outbreaks are rare, hence we assume that for the majority of time points $f(t)$ is 0, which implies that 
\begin{equation*}
    Y_t \mid \varepsilon_t \propto \mbox{Poisson}(\mu_t  \varepsilon_t )
\end{equation*}
and that, for example, for $N = T / 365$ years of daily data we would have $T$ data points to fit a model with at most N/7 + 7 + 4 + 1 parameters. For seven years of data this translates into 2,556 data points and 13 parameters. As a result, we can obtain highly precise estimates even in the presence of the extra dispersion introduced by $\varepsilon$. We therefore fit a Poisson Generalized Linear Model to regions that we know do not include natural disasters or outbreaks and, in the next step consider the MLE, $\hat{\alpha}(t)$, $\hat{w}(t)$, and $\hat{s}(t)$ as fixed.

In the second step we estimate $\Sigma$, and to do this we select a control contiguous region for which we know $f(t)=0$. To estimate the $\sigma^2$ and the autoregressive parameters we can use the law of total variance to note that the observed percent change from the mean
\begin{equation*}
    r_t = \frac{Y_t - \mu_t}{\mu_t}
\end{equation*}
has expected value and variance
\begin{equation*}
    \mathbb{E}\left(r_t \right) = 0 \mbox{ and } \mbox{Var}\left(r_t\right)= \sigma^2 + 1/\mu_t
\end{equation*}
Intuitively, these two components are the variance added by $\varepsilon$ and the Poisson variability respectively. The above implies that if we define the transformation:
\begin{equation*}
    Z_t = \frac{r_t}{\sqrt{\mbox{Var}\left(r_t\right)}}
\end{equation*}
we have an approximately normally distributed random variable with the same correlation structure as $\varepsilon_t$, which is what we want to estimate. We use the Yule-Walker equation approach to estimate the AR process parameters from the observed $Z_t$. To estimate $\sigma^2$ we use:
\begin{equation*}
    \hat{\sigma}^2 = \max\left\{ \frac{1}{T}\sum_{t=1}^T  \left(\frac{Y_t - \hat{\mu}_t}{\hat{\mu}_t}\right)^2  - \frac{1}{\mu_t}, \, 0\right\}
\end{equation*}
Using these estimates we can form an estimate $\hat{\Sigma}$ which we will consider known in the next step.

In the third step we treat $\hat{\Sigma}$ as known, assume $f(t)$ can be modeled as a natural cubic spline, and use iteratively re-weighted least squares to obtain an estimate $\hat{f}(t)$. To account for non-continuities due to direct effects from natural disasters we permit the spline to be discontinuous at the event day. 

\subsection{Approximation for uncorrelated data}
Note that our model can be easily adapted to be fitted to weekly data. In this case there is no need to include weekday effect parameters $w(t)$ and much less need for a correlation structure for the errors (Supplemental Figure \ref{supp-fig:weekly-correlated}). 

\subsection{Excess death estimates}
\label{subsec:excess-deaths}
We estimate cumulative excess deaths for any interval $I = [a, b]$ 
by adding the excess deaths for each day
in $I$: 
\begin{equation*}
    \mbox{Excess deaths} = \sum_{t = a}^b \hat{\mu}_t \hat{f}(t) 
\end{equation*}
We construct a 95\% confidence interval using the fact that if we define $H$ as the \emph{hat matrix} that gives us 
$\{ \hat{f}(a), \dots \hat{f}(b) \}^\top = H (Y_1, \dots, Y_T)^\top$ then

 \begin{equation*}
    \mbox{Var}\left[\sum_{t = a}^b \hat{\mu}_t \hat{f}(t) \right] = (\hat{\mu}_a,\dots,\hat{\mu}_b) H \hat{\Sigma} H^\top (\hat{\mu}_a,\dots,\hat{\mu}_b)^\top.
 \end{equation*}
 
\subsection{Simulation study}
To assess our procedure we conducted a Monte Carlo simulation. Specifically, we computed $\hat{\mu}_t$ for $t=1,\ldots, T$, using the Puerto Rico mortality data with model (\ref{eq:mean-model}) and generated data $Y_t^* \sim \mbox{Poisson}(\hat{\mu}_t)$ with serial correlation XXX EXPLAIN WHAT AR PROCESS YOU USED AND WHAT SIGMA YOU USED , THIS DESCRIPTION NEEDS TO BE PRECISE... the DATA IS NOT POISSON. XXX. We considered three scenarios: 1) a hurricane-like event where we expect mortality to increase rapidly and stay above normal levels for some time, 2) an epidemic-like event in which mortality increases slowly until reaching some plateau and then decreasing to normal levels, and 3) a time period of no event in which we expected our estimate of $f(t)$ to be centered around 0. Parameters describing our simulation are in \textbf{Supplemental Table XXX}. Our implementation produces accurate and precise results for both $f$ and $\Sigma$ (Supplemental Figure \ref{fig:simulation}).

\section{Discussion}
We have introduced methods and software useful for estimating excess mortality from weekly or daily counts. The engine of our approach is a statistical model that accounts for long-term trends, seasonal and day of the week effects, as well natural variation. An important contribution of this new approach are the improved uncertainty estimates permitted by modeling correlated errors, and 
the model accounts for natural variability with correlated errors, an empirically motivated assumption. Another advance is the use splines in a way that permits us to account for both direct and indirect effects. Users of our method should be aware that the choice of knots can change the results of the analysis and we implemented exploratory data analysis ensure that these choices were appropriate. We also note that the results of our analysis are also affected by the population size, $N_t$ which we are themselves estimates. 

We used the model to demonstrate that the indirect effects for Hurricane María on Puerto Rico were like nothing seen in seven previous hurricanes. We also demonstrated that the 2005 flu season and the 2014 Chikungunya epidemic had substnatial excess mortality in Puerto Rico. These results suggest a lack of robustness of the Puerto Rico's health system. In New Orleans in 2006, shortly after Katrina,
the U.S. Army Corps of Engineers admitted that faulty design specifications, incomplete sections,
and substandard construction of levee segments, contributed to the damage and a \$14.5 billion invested was made to construct stronger levees. In contrast, after Hurricane Georges resulted in similar excess mortality toll, as far as we know, no systematic effort was put in place to improve, for example, Puerto Rico’s electrical grid nor it's data collection infrastructure. On the contrary, it appears that the grid continued to deteriorate for the next 19 years. While Puerto Rico seems to have \emph{dodged a bullet} during the COVID-19 pandemic, Puerto Rico’s health system does not appear to be ready to withstand another shock to its fragile infrastructure.

Our analysis of CDC data demonstrated that by May 9, 2020 excess mortality was already above 100,000. We also performed a state-by-state analysis and found that over 58\% of these deaths occurred in New York (35,653) and three of its neighbors: New Jersey (14,537), Massachusetts (6,159), and Pennsylvania (8,383). Note that Connecticut, the other neighnoring state, was not included in the analysis due to missing data. It is important to note that the weekly death counts used in our analysis are adjusted for missing data by the CDC [CITE PAGE DESCRIBING XXX]. The accuracy of our results depends on the accuracy of this adjustment. 

Finally, we used our model to compare the effects of the pandemic in blacks and whites in cook county and found a disturbing trend, with elderly blacks affected much more severly than their white counterparts. Note that this analysis was not based on total mortality counts, but counts calculated from the Medical Examiner. Our conclusions assume that COVID-19 cases from whites and blacks are equally likely to be included in the Medical Examiner dataset. 

\bibliographystyle{unsrt}
\bibliography{ref.bib}

\newpage
\begin{center}
\LARGE{\textbf{Supplemental Material}}
\end{center}

% Supp figure 1: Trend and seasonal components
\begin{suppfigure}[ht]
	\begin{tabular}{lll}
	A&B\\
	\begin{subfigure}[t]{0.49\linewidth}
		\centering
		\includegraphics[width=1\linewidth]{figs/supp-figure-1a.pdf} 
	\end{subfigure}&
	\begin{subfigure}[t]{0.49\linewidth}
		\centering
		\includegraphics[width=1\linewidth]{figs/supp-figure-1b.pdf}
	\end{subfigure}&\\
    \end{tabular}
    \caption{Estimated components of the expected counts for six age groups in Puerto Rico. A) Death rate (deaths per 1,000 per year) trend components. B) Seasonal component for each age group as percentage increase or decrease from the average.}
    \label{supp-fig:model-components}
\end{suppfigure}

% Supp figure 2: F hat for all six hurricanes
\begin{suppfigure}[ht]
	\centering
	\includegraphics[width=1\linewidth]{figs/supp-figure-2.pdf} 
    \caption{Estimated hurricane effects as percent increase over expected mortality for the six hurricanes. The solid line corresponds to percent change from expected mortality and the shaded region represents a 95\% confidence interval.}
	\label{supp-fig:hurricane-effects}
\end{suppfigure}

% Supp figure 3: Fhat for age groups during chikungungya: NEED TO ADD FLU 2005
\begin{suppfigure}[ht]
	\begin{tabular}{lll}
	A&B\\
	\begin{subfigure}[t]{0.49\linewidth}
		\centering
		\includegraphics[width=1\linewidth]{figs/supp-figure-3a.pdf} 
	\end{subfigure}&
	\begin{subfigure}[t]{0.49\linewidth}
		\centering
		\includegraphics[width=1\linewidth]{figs/supp-figure-3b.pdf}
	\end{subfigure}&\\
    \end{tabular}
    \caption{Estimated event effects as percent increase over expected mortality. The solid line corresponds to percent change from expected mortality and the shaded region represents a 95\% confidence interval. A) Percent change during the Chikungunya epidemic in Puerto Rico for six age groups. B) Percent change during the 2005 flu season in Puerto Rico.}
	\label{supp-fig:chikungunya-agegroups}
\end{suppfigure}

% Supp figure 4: Top 12 jurisdiction most affected by covid 19 in USA
\begin{suppfigure}[ht]
	\centering
	\includegraphics[width=1\linewidth]{figs/supp-figure-4.pdf} 
	\caption{Estimated event effects as percent increase over expected mortality for New York City and 8 states in the United States with the largest percent increase in mortality rates during the COVID-19 pandemic. The solid line corresponds to percent change from expected mortality and the shaded region represents a 95\% confidence interval. Note that no data was provided for Connecticut or North Carolina.}
	\label{supp-fig:covid19-states}
\end{suppfigure}

% Supp figure 5: Excess deaths with 95 CI
\begin{suppfigure}[ht]
	\centering
	\includegraphics[width=1\linewidth]{figs/supp-figure-5.pdf} 
	\caption{Daily excess death estimates and 95\% confidence intervals for events in Puerto Rico starting on Sep 20, 2017, Sep 21, 1998, Aug 1, 2014, Mar 1, 2020, and Oct 1, 2004 for Hurricane Mar\'ia, Hurricane Georges, the Chikungunya epidemic, the COVID-19 pandemic, and the 2005 Flu season, respectively. The solid line corresponds to the cumulative excess death estimates and the shaded region represents a 95\% confidence interval.}
	\label{supp-fig:excess-deaths}
\end{suppfigure}

% Supp figure 6: Cook county effect by age groups
\begin{suppfigure}[ht]
	\centering
	\includegraphics[width=1\linewidth]{figs/supp-figure-6.pdf} 
	\caption{Estimated effect of the COVID-19 pandemic in Cook county, IL, as percent increase over expected mortality stratified by age groups. The solid line corresponds to percent change from expected mortality and the shaded region represents a 95\% confidence interval.}
	\label{supp-fig:cook-agegroups}
\end{suppfigure}

% Supp figure 7: QQ plots and ACF
\begin{suppfigure}[ht]
	\begin{tabular}{lll}
	A&B\\
	\begin{subfigure}[t]{0.49\linewidth}
		\centering
		\includegraphics[width=1\linewidth]{figs/supp-figure-7a.pdf} 
	\end{subfigure}&
	\begin{subfigure}[t]{0.49\linewidth}
		\centering
		\includegraphics[width=1\linewidth]{figs/supp-figure-7b.pdf}
	\end{subfigure}&\\
	C&D\\
	\begin{subfigure}[t]{0.49\linewidth}
		\centering
		\includegraphics[width=1\linewidth]{figs/supp-figure-7c.pdf}
	\end{subfigure}&
	\begin{subfigure}[t]{0.49\linewidth}
    	\centering
	    \includegraphics[width=1\linewidth]{figs/supp-figure-7d.pdf}
    \end{subfigure}\\
    \end{tabular}
    \caption{Evidence of correlated errors. A) A Poisson GLM was fitted to Puerto Rico daily death counts from an interval with no known natural disasters or outbreaks (Jan 1, 2006 to Dec 31, 2013). The plot shows the Pearson residuals quantiles versus theoretical quantiles from the normal distribution. One can see that the tail of the empirical data are larger than the theoretical values. B) The sample autocorrelation function for these Pearson residuals with the red-dash lines represent a 95\% confidence interval centered at zero. C) As A) but for residuals after prewhitening based on an estimate of the covariance matrix. D) As B) but for the prewhittened residuals.}
    \label{supp-fig:correlated-errors}
\end{suppfigure}

% Supp figure 8: Population displacement in Puerto Rico
\begin{suppfigure}[ht]
	\centering
	\includegraphics[width=1\linewidth]{figs/supp-figure-8.pdf} 
	\caption{Estimated population displacement in Puerto Rico after Hurricane Mar\'ia.}
	\label{supp-fig:pop-displacement-pr}
\end{suppfigure}


% Supp figure 9: Simulation study
\begin{suppfigure}[ht]
	\begin{tabular}{lll}
	A&B&C\\
	\begin{subfigure}[t]{0.30\linewidth}
		\centering
		\includegraphics[width=1\linewidth]{figs/supp-figure-9a.pdf} 
	\end{subfigure}&
	\begin{subfigure}[t]{0.30\linewidth}
		\centering
		\includegraphics[width=1\linewidth]{figs/supp-figure-9b.pdf}
	\end{subfigure}&
	\begin{subfigure}[t]{0.30\linewidth}
		\centering
		\includegraphics[width=1\linewidth]{figs/supp-figure-9c.pdf}
	\end{subfigure}\\
    \end{tabular}
    \caption{Simulation study: XXX NEEDS CAPTION XXX}
    \label{supp-fig:weekly-correlated}
\end{suppfigure}

\end{document}

